%%This is a very basic article template.
%%There is just one section and two subsections.
\documentclass{article}

\begin{document}


\title{\center{Data Networks project}}
\author{Group 7: \\ Sanjar Karaev, Igor Stassiy, Kirill Afanasev}
\date{\today}
\maketitle

\section{Introduction}
	The problem - design a protocol which provides service for delivering messages
	between hosts in multihost networks with different topologies. 
	
	Assumptions
		1)In the multihost networks hosts don't fail.
		2)The links have tendency to loose and corrupt messages from time to time.
		3)The links have varying capacities and MTU.
\section{Protocol Layers}
Our implementation consists of three layers - Application, Transport, Network
and Datalink. 

	Application layer API
		A purpose of this layer is in reading and writing messages into Cnet
		application layer. This layer handles Cnet_application_ready, performs
		Cnet_read_application and Cnet_write_application routines.
	
		init_application() - initialize the layer internal structures
		
		read_application(MSG message) - this function is used to read incoming message
		from transport to application layer
		
		
		
	Transport layer
		A purpose of this layer is in buffering incoming and outcoming messages
		before sending, handling sliding window paradigm, acknowledgements, resending
		lost and corrupted fragments.
		Top level API:
			init_transport() - initialize transport layer
		
			write_transport(CnetAddr destaddr,MSG message) - write outcoming message from
			application into transport layer
		
			read_transport(PACKET packet) - read incoming message from network to
			transport layer
		Internal functions:
			//TODO
		
		
	Network layer
		A purpose of this layer is in routing and discovering network, providing the
		Transport layer with fragmentation information.
		
		Top level API:
			init_network() - initialize network layer
		
			write_network(int link, PACKET packet) - write an outcoming message from
			transport to network layer
		
			int get_link_for_dest(CnetAddr destaddr) - detect a link for outcoming
			message
		
			int get_fragment_for_link(int link) - detect fragmentation size for the
			specified link
		
			read_network(link, PACKET packet) - read an incoming packet into network
			layer
		Internal functions:
			//TODO
		
		
	Datalink layer
		A purpose of this layer is in handling Cnet_write_physical and
		Cnet_physical_ready functions. 
		Top level API:
			init_datalink() - initialize datalink layer
		
			write_datalink(int link, PACKET packet) - write an outcoming packet into
			datalink layer
		Internal functions:
			//TODO



\section{Frame and Message Structures}

\subsection{Segmentation}

\subsection{Routing}

\section{Algorithms}

\section{Summary}

\end{document}

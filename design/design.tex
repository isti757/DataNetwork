%%This is a very basic article template.
%%There is just one section and two subsections.


\documentclass{article}

\usepackage{listings}

\begin{document}
\lstset{language=C}

\title{\center{Data Networks project}}
\author{Group 7: \\ Sanjar Karaev, Igor Stassiy, Kirill Afanasev}
\date{\today}
\maketitle

\section{Introduction}

	The problem - design a protocol which provides reliable service for
	in order delivering of messages between hosts in multihost networks with
	different topologies.
	
	Assumptions:
	\begin{enumerate}
	  \item In the multihost networks hosts don't fail.
	  \item The links have tendency to loose and corrupt frames from time to
	  time.
	  \item Frames can arrive out of order.
	  \item The links have varying capacities and MTU.
	  \item Nodes are only aware of their immediate neighbors at the beginning.
	\end{enumerate}
	
\section{Protocol Layers}
Our implementation consists of four layers - Application, Transport, Network
and Datalink. 

\begin{itemize}
  \item Application layer: \\
		The purpose of this layer is reading and writing messages into Cnet
		application layer. This layer handles Cnet\_application\_ready, performs
		Cnet\_read\_application and Cnet\_write\_application routines.
		
		Services provided by the layer:
		\begin{enumerate}
		  \item Forwarding messages to/from the application layer 
		\end{enumerate}
	
	Top level API:
	\begin{verbatim}
    // initialize the layer internal structures
    void init_application()
		
    // reads incoming message from transport to application layer
    void read_to_application(MSG message) 
    
    // write the message up to the application
    void write_from_application(CnetAddr addr, MSG msg)    
	\end{verbatim}
		
\item	Transport layer: \\
		The purpose of this layer is in buffering incoming and outcoming messages
		before sending, handling sliding window paradigm, acknowledgements, resending
		lost and corrupted fragments.
		
		Services provided by the layer:
    \begin{enumerate}
      \item Reliably delivers a packet from one host to another, avoiding packet
      loss and signaling if delivery is not possible.
      \item Segments the packets to send according to the link MTU.
    \end{enumerate}
		
		Top level API:
  \begin{verbatim}
    // initialize transport layer
    void init_transport()
		
    // read incoming message from network to transport layer
    void read_to_transport(PACKET pkt)
    
    // write outcoming message from application into transport layer
    void write_from_transport(CnetAddr addr, MSG msg) 
  \end{verbatim}
		
		
\item Network layer: \\
		The purpose of this layer is in routing and discovering network, providing the
		Transport layer with fragmentation information.
		
		Services provided by the layer:
    \begin{enumerate}
      \item Determines the routing information for a host, by keeping a routing
      table.
      \item Determines lowest MTU on a path from current host to another.
    \end{enumerate}
		
    Top level API:
		
  \begin{verbatim}
    // initialize network layer
    void init_network()	
   		
    // detect a link for outcoming message
    int get_next_link_for_dest(CnetAddr destaddr)
		
    // detect fragmentation size for the specified link
    int get_mtu_for_link(int link) 
		
    // read an incoming packet into network layer
    void read_to_network(link, PACKET packet)
    
     // write an outgoing message from transport to network layer
    void write_from_network(int link, PACKET packet)
    
  \end{verbatim}
		

\item	Datalink layer: \\
		The purpose of this layer is in handling Cnet\_write\_physical and
		Cnet\_physical\_ready functions. 
		
		Services provided by the layer:
    \begin{enumerate}
      \item Determines the validity of the received datagram.
      \item Sends a datagram to the link.
    \end{enumerate}
		
		Top level API:

  \begin{verbatim}
    // initialize datalink layer
    void init_datalink()
    
    // write an incoming frame into datalink layer
    void read_to_datalink(int link, FRAME frame)
		
    // write an outcoming packet into datalink layer
    void write_datalink(int link, PACKET packet)
  \end{verbatim}

\end{itemize}

\section{Frame and Message Structures}

\subsection{Segmentation}

\subsection{Routing}

\section{Algorithms}

\section{Summary}

\end{document}

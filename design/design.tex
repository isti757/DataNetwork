%%This is a very basic article template.
%%There is just one section and two subsections.


\documentclass{article}

\usepackage{listings}

\begin{document}
\lstset{language=C}

\title{\center{Data Networks project}}
\author{Group 7: \\ Sanjar Karaev, Igor Stassiy, Kirill Afanasev}
\date{\today}
\maketitle

\section{Introduction}

	The problem - design a protocol which provides reliable service for
	in order delivering of messages between hosts in multihost networks with
	different topologies.
	
	Assumptions:
	\begin{enumerate}
	  \item In the multihost networks hosts don't fail.
	  \item The links have tendency to loose and corrupt frames from time to
	  time.
	  \item Frames can arrive out of order.
	  \item The links have varying capacities and MTU.
	  \item Nodes are only aware of their immediate neighbors at the beginning.
	\end{enumerate}
	
\section{Protocol Layers}
Our implementation consists of four layers - Application, Transport, Network
and Datalink. 

\begin{itemize}
  \item Application layer: \\
		The purpose of this layer is reading and writing messages into Cnet
		application layer. This layer handles Cnet\_application\_ready, performs
		Cnet\_read\_application and Cnet\_write\_application routines.
		
		Services provided by the layer:
		\begin{enumerate}
		  \item Forwarding messages to/from the application layer 
		\end{enumerate}
	
	Top level API:
	\begin{verbatim}
    // initialize the layer internal structures
    void init_application()
		
    // reads incoming message from transport to application layer
    void read_to_application(CnetAddr addr, MSG message, int len) 
    
    // write the message up to the application
    void write_from_application(CnetAddr addr, PACKET pkt)    
	\end{verbatim}
		
\item	Transport layer: \\
		The purpose of this layer is in buffering incoming and outcoming
		 messages before sending, handling sliding window paradigm, 
		 acknowledgements, resending lost and corrupted fragments.
		
		Services provided by the layer:
    \begin{enumerate}
      \item Reliably delivers a packet from one host to another, avoiding
       packet loss and signaling if delivery is not possible.
      \item Segments the packets to send according to the link MTU.
    \end{enumerate}
		
		Top level API:
  \begin{verbatim}
    // initialize transport layer
    void init_transport()
		
    // read incoming message from network to transport layer
    void read_transport(CnetAddr destaddr, PACKET pkt);
    
   // write outcoming message from application into transport layer
extern void write_transport(PACKET pkt, CnetAddr dest);
		
		
\item Network layer: \\
		The purpose of this layer is in routing and discovering network,
		 providing the Transport layer with fragmentation information.
		
		Services provided by the layer:
    \begin{enumerate}
      \item Determines the routing information for a host, by keeping a routing
      table.
      \item Determines lowest MTU on a path from current host to another.
    \end{enumerate}
		
    Top level API:
		
  \begin{verbatim}
    // initialize network layer
    void init_network()	
   		
   // detect a link for outcoming message
extern int get_next_link_for_dest(CnetAddr destaddr);
		
   // detect a link for outcoming message
extern int get_next_link_for_dest(CnetAddr destaddr);

// detect fragmentation size for the specified link
extern int get_mtu_for_link(int link);

// read an incoming packet into network layer
extern void read_network(int link, DATAGRAM dtg);

// write an incoming message from datalink to network layer
extern void write_network(int link, DATAGRAM dtg);
    
  \end{verbatim}
		

\item	Datalink layer: \\
		The purpose of this layer is in handling Cnet\_write\_physical and
		Cnet\_physical\_ready functions. 
		
		Services provided by the layer:
    \begin{enumerate}
      \item Determines the validity of the received datagram.
      \item Sends a datagram to the link.
    \end{enumerate}
		
		Top level API:

  \begin{verbatim}
   // initialize datalink layer
extern void init_datalink();

// write an incoming frame into datalink layer
extern void read_datalink(int link, DATAGRAM dtg);
  \end{verbatim}

\end{itemize}

\section{Message Packet Datagram and Frame Structures}
We define the following basic transfer units: Message, Packet, Datagram, and
Frame. Every unit encapsulates data, which is the unit from the layer above and
header, which is the information added by the current layer. 

\begin{itemize}
  \item Message: \\
  Basic transfer unit of application layer. 
  
  //Data:  an array of char received from cnet application layer
  
  //Header:
  \item Packet: \\
  Basic transfer unit of transport layer. 
  
  //Data:  a message of application layer
  
  //Header:  
  \begin{itemize}
    \item len - the length of the message
    \item dest - the destination address
    \end{itemize}
  
  \item Datagram: \\
   Basic transfer unit of network layer
   
   //Data: a packet of transport layer
   
   //Header: 
   \begin{itemize}
     \item src (source address)
     \item dest (destination address)
     \item kind (it stands for whether the message is a data 
     or an acknowledgement)
     \item seqno (sequence number)
     \item hopsleft (indicates how many times the datagram 
     can still be sent across a link)
     \item hopstaken  (indicates how many times it has already been sent)
    \item timesent (indicates at what time the datagram was sent)
   \item checksum (serves for the detection of possible data corruption)
\end{itemize}

\item Frame: \\

 Basic transfer unit of Datalink layer
 
 //Data: a datagram of network layer
 //Header:
 \end{itemize}

\subsection{Segmentation}

\subsection{Routing}

\section{Algorithms}

\section{Summary}

\end{document}
